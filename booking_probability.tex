% Options for packages loaded elsewhere
\PassOptionsToPackage{unicode}{hyperref}
\PassOptionsToPackage{hyphens}{url}
\documentclass[
]{article}
\usepackage[margin=1in]{geometry}
\usepackage{xcolor}
\usepackage{amsmath,amssymb}
\setcounter{secnumdepth}{-\maxdimen} % remove section numbering
\usepackage{iftex}
\ifPDFTeX
  \usepackage[T1]{fontenc}
  \usepackage[utf8]{inputenc}
  \usepackage{textcomp} % provide euro and other symbols
\else % if luatex or xetex
  \usepackage{unicode-math} % this also loads fontspec
  \defaultfontfeatures{Scale=MatchLowercase}
  \defaultfontfeatures[\rmfamily]{Ligatures=TeX,Scale=1}
\fi
\usepackage{lmodern}
\ifPDFTeX\else
  % xetex/luatex font selection
\fi
% Use upquote if available, for straight quotes in verbatim environments
\IfFileExists{upquote.sty}{\usepackage{upquote}}{}
\IfFileExists{microtype.sty}{% use microtype if available
  \usepackage[]{microtype}
  \UseMicrotypeSet[protrusion]{basicmath} % disable protrusion for tt fonts
}{}
\makeatletter
\@ifundefined{KOMAClassName}{% if non-KOMA class
  \IfFileExists{parskip.sty}{%
    \usepackage{parskip}
  }{% else
    \setlength{\parindent}{0pt}
    \setlength{\parskip}{6pt plus 2pt minus 1pt}}
}{% if KOMA class
  \KOMAoptions{parskip=half}}
\makeatother
\usepackage{graphicx}
\makeatletter
\newsavebox\pandoc@box
\newcommand*\pandocbounded[1]{% scales image to fit in text height/width
  \sbox\pandoc@box{#1}%
  \Gscale@div\@tempa{\textheight}{\dimexpr\ht\pandoc@box+\dp\pandoc@box\relax}%
  \Gscale@div\@tempb{\linewidth}{\wd\pandoc@box}%
  \ifdim\@tempb\p@<\@tempa\p@\let\@tempa\@tempb\fi% select the smaller of both
  \ifdim\@tempa\p@<\p@\scalebox{\@tempa}{\usebox\pandoc@box}%
  \else\usebox{\pandoc@box}%
  \fi%
}
% Set default figure placement to htbp
\def\fps@figure{htbp}
\makeatother
\setlength{\emergencystretch}{3em} % prevent overfull lines
\providecommand{\tightlist}{%
  \setlength{\itemsep}{0pt}\setlength{\parskip}{0pt}}
\usepackage{bookmark}
\IfFileExists{xurl.sty}{\usepackage{xurl}}{} % add URL line breaks if available
\urlstyle{same}
\hypersetup{
  hidelinks,
  pdfcreator={LaTeX via pandoc}}

\author{}
\date{}

\begin{document}

\section{A Structured approach to Quotation pricing in Container
Shipping}\label{a-structured-approach-to-quotation-pricing-in-container-shipping}

Container‑shipping rate setting often looks reactive and chaotic: market
swings, inconsistent customer behavior, and noisy acceptance/rejection
outcomes. This note does not attempt to solve pricing end‑to‑end, but
instead isolates and studies a simple mathematical core of the problem:
how quote requests and candidate rates interact through accept/reject
decisions.

We treat this core as a small case study in applied category theory and
order theory. The aim is to introduce basic categorical notions---such
as isomorphisms, epimorphisms, and adjunctions---through a concrete
problem in the logistics sector. The central object is a feasibility
relation that records, for each request--rate pair, whether the quote
would be accepted or rejected, and we use it to introduce basic notions
such as enriched profunctors, nuclei, and Galois adjunctions in a
concrete setting. Familiar constructs---willingness‑to‑pay curves,
rejection and booking curves, and suggested selling rates---then appear
as derived objects of this single relation once we impose a bit of
structure.

\begin{center}\rule{0.5\linewidth}{0.5pt}\end{center}

\subsection{1. Market Requests as a Total
Order}\label{market-requests-as-a-total-order}

We begin with the set of all market quote requests:

\[
Q_{\text{market}} = \{q_{(1)}, q_{(2)}, \dots, q_{(N)}\}.
\]

We assume that \textbf{these requests can be totally ordered}:

\[
q_{(1)} \le q_{(2)} \le \dots \le q_{(N)}.
\]

\subsubsection{What does this order
represent?}\label{what-does-this-order-represent}

Many things in pricing behave monotonically:

\begin{itemize}
\tightlist
\item
  shipment volume\\
\item
  service urgency/flexibility\\
\item
  customer segment\\
\item
  historical behavior\\
\item
  any internal ``flexibility'' or ``value'' score
\end{itemize}

As long as \emph{higher} in the order means \emph{more willing to pay},
the model works.

\begin{center}\rule{0.5\linewidth}{0.5pt}\end{center}

\subsection{2. Market Rates Respect the Same
Ordering}\label{market-rates-respect-the-same-ordering}

We similarly assume the market has an ordered set of rates:

\[
R_{\text{market}}.
\]

There is a one-to-one \textbf{monotone map}:

\[
m : Q_{\text{market}} \to R_{\text{market}}
\]

assigning each request the ``market-clearing'' rate.

\subsubsection{Meaning of monotonicity}\label{meaning-of-monotonicity}

If one request is less sensitive and sits higher in the order, the rate
assigned to it must not be lower than that of a more sensitive request.

\begin{center}\rule{0.5\linewidth}{0.5pt}\end{center}

\subsection{3. Acceptance/Rejection as a Boolean Feasibility
Relation}\label{acceptancerejection-as-a-boolean-feasibility-relation}

Because requests and rates share the same ordering, any quote request
can be paired with any candidate market rate and classified: either the
offered rate sits below that request's clearing level and is accepted,
or it exceeds it and is rejected. Acceptance occurs when the offered
rate does not exceed the clearing rate of the request.

\[
r \le m(q).
\]

Formally, treat this as a feasibility relation: it is antitone in the
rate (higher rates are harder to accept) and monotone in the request
(less-sensitive customers accept more). Using the opposite order on
rates makes the relation monotone in both arguments:

\[
F : R_{\text{market}}^{op} \times Q_{\text{market}}
\to \mathbf{Bool},
\qquad
F(r,q) = (r\le m(q)).
\]

This is the fundamental decision rule; rejection happens exactly when
\(F(r,q)\) is false.

\begin{center}\rule{0.5\linewidth}{0.5pt}\end{center}

\subsection{\texorpdfstring{4. Quantile Embeddings: Putting Everything
on
\([0,1]\)}{4. Quantile Embeddings: Putting Everything on {[}0,1{]}}}\label{quantile-embeddings-putting-everything-on-01}

To compare requests and rates cleanly, embed any finite total order
\(Q\) into \([0,1]\) by turning each element into its cumulative rank:
count how many elements are at or below it, then divide by \(|Q|\).

Define the quantile embedding

\[
\phi_Q : Q \to [0,1],
\qquad
\phi_Q(q)
= \frac{|\{q' \in Q : q' \le q\}|-1}{|Q|-1}.
\]

This map is monotone and bijective onto its image, so it is an
order‑isomorphism:
\[
Q \cong [0,1]_Q.
\]
\[
[0,1]_Q := \mathrm{im}(\phi_Q) \subseteq [0,1].
\]
We will freely identify elements of \(Q\) with their images in
\([0,1]_Q\) and read \(\phi_Q(q)\) as the standardized rank of \(q\) on
the \([0,1]\) scale.

\begin{figure}
\centering
\pandocbounded{\includegraphics[keepaspectratio,alt={Quantile embedding of a finite totally ordered set Q into {[}0,1{]}}]{figures/quantile_embedding.jpg}}
\caption{Quantile embedding of a finite totally ordered set \(Q\) into
\([0,1]\)}
\end{figure}

\subsubsection{Request quantiles}\label{request-quantiles}

Applied to market requests (take \(Q = Q_{\text{market}}\)):

\[
\phi_Q : Q_{\text{market}} \to [0,1],
\qquad
\phi_Q(q) = \frac{|\{q' \in Q_{\text{market}} : q' \le q\}|-1}{|Q_{\text{market}}|-1}
= \frac{k-1}{N-1} \text{ for } q = q_{(k)}.
\]

Every request gets a rank between 0 and 1. We write

\[
[0,1]_Q := \mathrm{im}(\phi_Q) \subseteq [0,1]
\]

for the resulting request‑quantile axis inside the unit interval.

\subsubsection{Rate quantiles}\label{rate-quantiles}

\[
\phi_R : R_{\text{market}} \to [0,1].
\]

For rates, the same construction with \(P = R_{\text{market}}\) gives:

\[
\phi_R(r) = \frac{|\{r' \in R_{\text{market}} : r' \le r\}|-1}{|R_{\text{market}}|-1}.
\]

We similarly denote the image of \(\phi_R\) by

\[
[0,1]_R := \mathrm{im}(\phi_R) \subseteq [0,1],
\]

the rate‑quantile axis inside the unit interval.

Because \(m\) is monotone and preserves ranks, \(\phi_R \circ m\) and
\(\phi_Q\) coincide, so \(Q_{\text{market}}\) and \(R_{\text{market}}\)
are isomorphic as total orders via their quantile images \([0,1]_Q\) and
\([0,1]_R\) (both sitting inside the same unit interval).

\[
\phi_R(m(q_{(k)})) = \phi_Q(q_{(k)}).
\]

\subsubsection{Intuition}\label{intuition}

\begin{itemize}
\tightlist
\item
  Now every request and every rate corresponds to a \emph{single number}
  in \([0,1]\).\\
\item
  They live on the same axis; we use subscripts such as \([0,1]_Q\) and
  \([0,1]_R\) only when we want to remember whether a quantile came from
  requests or from rates.\\
\item
  Higher quantile = higher willingness to pay / higher offered rate.
\end{itemize}

From here on we work entirely in quantile coordinates on \([0,1]\),
reusing \(r\) and \(q\) to denote rate and request quantiles. To recover
physical values, apply the inverses

\[
\phi_R^{-1} : [0,1]_R \to R_{\text{market}},
\qquad
\phi_Q^{-1} : [0,1]_Q \to Q_{\text{market}},
\]

when needed.

\begin{center}\rule{0.5\linewidth}{0.5pt}\end{center}

\subsection{5. Closures, Structural WTP, and the Rejection
Adjunction}\label{closures-structural-wtp-and-the-rejection-adjunction}

\subsubsection{Closures, the nucleus, and structural
WTP}\label{closures-the-nucleus-and-structural-wtp}

In quantile space, the market feasibility relation becomes a
\(\mathbf{Bool}\)-enriched \textbf{profunctor} between the ordered
quantile spaces of rates and requests:

\[
F : [0,1]_R^{op} \times [0,1]_Q \to \mathbf{Bool},
\qquad
F(r,q) = \mathbf{true} \; \Longleftrightarrow \; r \le q.
\]

This single feasibility relation induces an adjunction between the
powerset lattices \(\mathcal{P}([0,1])\) of rates and requests via two
dual closure operators: - The \textbf{upper closure} gathers all rates
that a family of requests would accept. - The \textbf{lower closure}
gathers all requests that would accept a family of rates.

Each is a monotone endomap on \(\mathcal{P}([0,1])\); they are the two
sides of a Galois connection from which both willingness-to-pay (WTP)
and expected rejection levels will be derived as adjoint scalar maps.

\subsubsection{Upper closure}\label{upper-closure}

\[
F^\sharp : \mathcal{P}([0,1]) \to \mathcal{P}([0,1]), \qquad
F^\sharp(B)=\{ r \in [0,1] : \forall q\in B,\, F(r,q)\}.
\]

For a singleton \(B=\{q\}\) this reduces to
\(F^\sharp(\{q\})=\{ r \in [0,1] : r \le q \}\); when no confusion
arises, we abbreviate this as \(F^\sharp(q)\).

In total orders, each downward-closed set has a maximum and each
upward-closed set has a minimum. This lets us collapse the set-valued
closure into an \textbf{extremal map}: \[
r^* : [0,1] \to [0,1], \quad r^*(q)\;:=\;\max F^\sharp(q)
\]

where \(r^*(q)\) stands for the largest rate still acceptable for
request quantile \(q\), or structural Willingless-To-Pay (WTP).

Visualising the \((\phi_R, \phi_Q)\) unit square:

\pandocbounded{\includegraphics[keepaspectratio,alt={WTP for a given request quantile}]{figures/wtp_for_q.jpg}}

In the figure, the diagonal \(x=y\) splits the unit square: points
below it (blue-to-red side) are rejections (\(q<r\)), points above it
are acceptances. - Each quotation request corresponds to a horizontal
line at its quantile \(q\). - The WTP \(r^*(q)\) is the intersection
with the diagonal \(x=y\); projecting down gives the maximum feasible
rate quantile. - Feasible rates \(F^\sharp(q)\) sit on or below that
line (to the left of the intersection). - Moving the horizontal line
upward (higher \(q\)) shifts the WTP rightward (higher rate quantile).

\subsubsection{Lower closure}\label{lower-closure}

\[
F_\flat : \mathcal{P}([0,1]) \to \mathcal{P}([0,1]), \qquad
F_\flat(A)=\{ q \in [0,1] : \forall r\in A,\, F(r,q)\}.
\]

For a singleton \(A=\{r\}\) this gives
\(F_\flat(\{r\})=\{ q \in [0,1] : r \le q \}\), which we abbreviate as
\(F_\flat(r)\). We then define:

\[
q^* : [0,1] \to [0,1], \quad q^*(r)\;:=\;\min F_\flat(r),
\]

where \(q^*(r)\) is the least quotation request quantile that will still
accept rate quantile \(r\). This corresponds to structural rejection
probability.

\begin{figure}
\centering
\pandocbounded{\includegraphics[keepaspectratio,alt={Market feasibility relation in quantile space; rejections lie below the diagonal, acceptances above}]{figures/market_feasibility_relation.jpg}}
\caption{Market feasibility relation in quantile space; rejections lie
below the diagonal, acceptances above}
\end{figure}

\begin{itemize}
\tightlist
\item
  Dually from Upper closure, for a fixed rate quantile \(r\), the
  booking threshold \(q^*(r)\) is the intersection of the vertical line
  at \(r\) with the diagonal; projecting left gives the least request
  quantile that will accept that rate.
\item
  If a rate sits at quantile 0.3, then 30\% of requests sit below it →
  they reject.\\
\item
  The market rejection curve is a straight line with slope 1 (the
  booking curve is its complement, \(1-r\)).
\end{itemize}

\subsubsection{The nucleus}\label{the-nucleus}

Working in quantile coordinates, we view the feasibility relation as a
\(\mathbf{Bool}\)‑enriched profunctor

\[
F : [0,1]_R^{op} \times [0,1]_Q \to \mathbf{Bool},
\]

where we implicitly identify market rates and requests with their
quantiles via \(\phi_R\) and \(\phi_Q\). The \textbf{nucleus} of \(F\)
is the subset

\[
\mathrm{Nuc}(F) \;\subseteq\; \mathcal{P}([0,1]_R) \times \mathcal{P}([0,1]_Q)
\]

of pairs \((A,B)\) of rate‑ and request‑quantile sets defined by

\[
\mathrm{Nuc}(F) = \{(A,B) :\; A = F^\sharp(B),\; B = F_\flat(A)\},
\]

with \(A \subseteq [0,1]_R\) a set of rate quantiles and
\(B \subseteq [0,1]_Q\) a set of request quantiles. Such pairs are ``as
tight as possible'' with respect to \(F\): \(A\) is exactly the set of
rates supported by \(B\), and \(B\) is exactly the set of requests
supporting \(A\).

The extremal maps introduced above

\[
r^* : [0,1]_Q \to [0,1]_R, \quad r^{*}(q) := \max F^\sharp(\{q\}),
\]

\[
q^* : [0,1]_R \to [0,1]_Q, \quad q^{*}(r) := \min F_\flat(\{r\}),
\]

extract these interval endpoints. Every nucleus pair can be encoded
either by its request endpoint or by its rate endpoint:

\[
\mathrm{Nuc}(F) =\{([0,r^*(q)],[q,1]) : q \in [0,1]_Q\}
\]

\[
\mathrm{Nuc}(F) =\{([0,r],[q^*(r),1]) : r \in [0,1]_R\}
\]

Equivalently, we may parameterize nucleus pairs just by their extremal
points:

\[
\mathrm{Nuc}(F)_{\mathrm{ext}}=\{(r,q) \in [0,1]_R \times [0,1]_Q : r = r^*(q),\; q = q^*(r)\}.
\]

\pandocbounded{\includegraphics[keepaspectratio,alt={Market nucleus points in (r,q) space}]{figures/nucleus_points.jpg}}
- A generic off-diagonal point \((r,q)\) is mapped horizontally to
\((r^*(q),q)\) and vertically to \((r,q^*(r))\), illustrating how WTP
and rejection are extracted as extremal summaries of the feasibility
relation. - Points on the diagonal where \(r=r^*(q)\) and \(q=q^*(r)\)
are fixed by these mappings; these diagonal points are exactly the
elements of \(\mathrm{Nuc}(F)_{\mathrm{ext}}\). - The figure thus shows
how the nucleus picks out a one‑dimensional set of extremal points while
arbitrary \((r,q)\) pairs are mapped onto this WTP/rejection
correspondence.

\subsection{6. Adjoint scalar maps, WTP, and
rejection}\label{adjoint-scalar-maps-wtp-and-rejection}

The scalar maps \(r^*\) and \(q^*\) summarize the nucleus at the level
of individual quantiles.

\begin{itemize}
\item
  The \textbf{WTP map} \(r^* : [0,1]_Q \to [0,1]_R\) answers:
  \textgreater{} For a given quotation request quantile \(q\), what is
  the largest rate quantile that this request will still accept?

  Order‑theoretically, the feasible rate set \(F^\sharp(q)\) is an
  initial segment of the rate‑quantile line, and \(r^*(q)\) is simply
  its right endpoint: the supremum rate that remains acceptable for
  requests at level \(q\).
\item
  The \textbf{rejection map} \(q^* : [0,1]_R \to [0,1]_Q\) answers the
  dual question: \textgreater{} For a given rate quantile \(r\), what is
  the lowest quotation‑request quantile that will accept this rate?

  Dually, the feasible request set \(F_\flat(r)\) is a terminal segment
  of the request‑quantile line, and \(q^*(r)\) is its left endpoint: the
  infimum request quantile that still accepts the rate level \(r\).
\end{itemize}

Given a uniform distribution of request quantiles on \([0,1]_Q\), this
same map \(q^*\) explains why we call it ``rejection'': the fraction of
requests that reject at rate \(r\) is exactly the mass of quantiles
below \(q^*(r)\), which by uniformity equals \(q^*(r)\) itself. In the
fully symmetric normalization used here we have \(q^*(r)=r\), so the
rejection probability at rate quantile \(r\) is \(r\) and the booking
probability is \(1-r\).

The pointwise nucleus \(\mathrm{Nuc}(F)_{\mathrm{ext}}\) packages the
dual correspondence between these two scalar summaries: each pair
\((r,q)\) with \(r=r^*(q)\) and \(q=q^*(r)\) links a structural WTP
level \(r\) to the rejection quantile \(q\) it induces. Composing the
WTP map with the rejection map makes this correspondence explicit:

\[
q^*(r^*(q)) = q.
\]

In words: at the structural willingness‑to‑pay level of a request at
quantile \(q\), the market rejection probability is precisely \(q\)
itself (and the corresponding booking probability is \(1-q\)). WTP
levels and rejection probabilities are thus two views of the same
nucleus‑induced correspondence.

\begin{center}\rule{0.5\linewidth}{0.5pt}\end{center}

\subsection{7. A Carrier's Customer Subset as a Restricted
Nucleus}\label{a-carriers-customer-subset-as-a-restricted-nucleus}

So far we have described the market as a single feasibility profunctor

\[
F : [0,1]_R^{op} \times [0,1]_Q \to \mathbf{Bool}
\]

with nucleus \(\mathrm{Nuc}(F)\) and extremal scalar maps

\[
r^* : [0,1]_Q \to [0,1]_R,
\qquad
q^* : [0,1]_R \to [0,1]_Q,
\]

encoding structural willingness‑to‑pay and the market rejection map. The
corresponding pointwise nucleus records these maps at the level of
extremal pairs:

\[
\mathrm{Nuc}(F)_{\mathrm{ext}}
= \{(r,q) \in [0,1]_R \times [0,1]_Q : r = r^*(q),\; q = q^*(r)\}.
\]

In practice, however, an individual carrier only sees a subset of the
market's quote requests.

\subsubsection{Restricting the request
side}\label{restricting-the-request-side}

Let \(Q_{\text{carrier}}\subseteq Q_{\text{market}}\) be the subset of
market requests that actually appear in the carrier's portfolio.
Applying the same quantile construction as in section 4 to
\(Q_{\text{carrier}}\) produces a carrier request‑quantile subset

\[
[0,1]_{Q_{\text{carrier}}} \subseteq [0,1]
\]

as the image of the carrier quantile embedding

\[
\phi_{Q_{\text{carrier}}} : Q_{\text{carrier}} \to [0,1],
\qquad
\phi_{Q_{\text{carrier}}}(q)
=
\frac{
|\{q' \in Q_{\text{carrier}} : q' \le q\}|-1
}{
|Q_{\text{carrier}}|-1
},
\]

and recalling that the market embedding
\(\phi_Q : Q_{\text{market}} \to [0,1]\) has image
\([0,1]_Q\subseteq[0,1]\), the inclusion
\(Q_{\text{carrier}}\hookrightarrow Q_{\text{market}}\) induces a
monotone embedding between the corresponding request‑quantile images

\[
j : [0,1]_{Q_{\text{carrier}}} \hookrightarrow [0,1]_Q,
\]

which we read as ``carrier request quantiles included into market
request quantiles''. Because these are totally ordered sets, this
inclusion sits in a Galois pair of adjoints. The \textbf{left adjoint}

\[
j_! : [0,1]_Q \to [0,1]_{Q_{\text{carrier}}},
\]

is characterized by

\[
j_!(q) \le q_c
\;\Longleftrightarrow\;
q \le j(q_c)
\quad
\text{for all } q \in [0,1]_Q,\; q_c \in [0,1]_{Q_{\text{carrier}}}.
\]

On the chain \([0,1]_{Q_{\text{carrier}}}\) this becomes the
\textbf{least} carrier quantile whose image lies above \(q\):

\[
j_!(q)
=
\min \{ q_c \in [0,1]_{Q_{\text{carrier}}} : q \le j(q_c)\}
=
1-
\frac{
|\{ q_c \in [0,1]_{Q_{\text{carrier}}} : q \le j(q_c)\}|-1
}{
|Q_{\text{carrier}}|-1
},
\]

whenever such a minimum exists. In a more familiar numerical setting,
this plays the role of a ``ceiling'' operation: given an embedded value
\(q\), \(j_!(q)\) picks the smallest carrier quantile that does not lie
below it. Because \(j\) is an order-embedding, the composite acts as the
identity on carrier quantiles:

\[
j_!\bigl(j(q_c)\bigr) = q_c
\quad
\text{for all } q_c \in [0,1]_{Q_{\text{carrier}}}.
\]

\pandocbounded{\includegraphics[keepaspectratio,alt={Left adjoint sending market quantiles down to carrier quantiles}]{figures/left_adjoint.jpg}}
- The inclusion \(j\) is drawn in green and sends carrier quantiles
\(q_c\) into the market quantile set \([0,1]_Q\). - The left adjoint
\(j_!\) is drawn in red and sends each market quantile \(q\)
\textbf{down} to the least carrier quantile \(q_c\) whose image lies at
or above \(q\). - In terms of order, \(j_!(q)\) is the best carrier-side
approximation \emph{from below} to \(q\) that respects the inclusion
\(j\).

Symmetrically, the \textbf{right adjoint}

\[
\overline{j} : [0,1]_Q \to [0,1]_{Q_{\text{carrier}}},
\]

is characterized by

\[
j(q_c) \le q
\;\Longleftrightarrow\;
q_c \le \overline{j}(q)
\quad
\text{for all } q \in [0,1]_Q,\; q_c \in [0,1]_{Q_{\text{carrier}}}.
\]

Concretely, \(\overline{j}(q)\) is the largest carrier quantile that
still maps below (or equal to) the market quantile \(q\); in our finite
setting

\[
\overline{j}(q)
=
\max \{ q_c \in [0,1]_{Q_{\text{carrier}}} : j(q_c) \le q\}
=
\frac{
|\{q_c \in [0,1]_{Q_{\text{carrier}}} : j(q_c) \le q\}|
}{
|Q_{\text{carrier}}|
}.
\]

Both adjoints will be useful: in what follows we use the right adjoint
\(\overline{j}\) to define the carrier rejection map
\(q^*_{\text{carrier}}\) from the market rejection map \(q^*\), and
later constructions can equally be phrased in terms of the left adjoint
\(j_!\) when pushing carrier quantiles forward into market order.

\begin{figure}
\centering
\pandocbounded{\includegraphics[keepaspectratio,alt={Carrier subset pulled back into market quantiles}]{figures/subset_precomposition.jpg}}
\caption{Carrier subset pulled back into market quantiles}
\end{figure}

Categorically, we restrict the market profunctor along the inclusion
\(j\) on the request side:

\[
F_{\text{carrier}}(r,q)
:=
F\bigl(r,\,j(q)\bigr),
\qquad
F_{\text{carrier}} : [0,1]_R^{op} \times [0,1]_{Q_{\text{carrier}}} \to \mathbf{Bool}.
\]

This is the carrier's feasibility relation: it records booking behavior
only for those requests that actually appear in the carrier's portfolio.

\pandocbounded{\includegraphics[keepaspectratio,alt={Carrier feasibility relation in quantile space}]{figures/carrier_feasibility_relation.jpg}}
- Above is the smoothed version of this restriction, which is
approximately how the curve would look if the market pool and the
carrier subset were very large - The band still monotone‑separates
accept/reject regions, defining the carrier's feasibility frontier. -
The feasibility boundary shifts off \(x=y\) compared to the market
because the carrier sees a different mix; acceptance at a given rate
quantile can be above or below market.

\subsubsection{The carrier nucleus as a slice of the market
nucleus}\label{the-carrier-nucleus-as-a-slice-of-the-market-nucleus}

The restricted profunctor \(F_{\text{carrier}}\) has its own nucleus

\[
\mathrm{Nuc}(F_{\text{carrier}})
\subseteq
\mathcal{P}([0,1]_R) \times \mathcal{P}([0,1]_{Q_{\text{carrier}}}),
\]

consisting of pairs \((A,B_{\text{c}})\) of rate‑ and carrier‑request
sets satisfying

\[
A = F_{\text{carrier}}^\sharp(B_{\text{c}}),
\qquad
B_{\text{c}} = F_{\text{carrier}}^\flat(A).
\]

By construction, this nucleus is just the market nucleus seen through
the inclusion \(j\): a pair \((A,B_{\text{c}})\) belongs to
\(\mathrm{Nuc}(F_{\text{carrier}})\) exactly when

\[
\bigl(A,\, j(B_{\text{c}})\bigr) \in \mathrm{Nuc}(F).
\]

In other words, we obtain the carrier nucleus by restricting the market
nucleus on the request side to those subsets that live entirely inside
\(Q_{\text{carrier}}\). All the structure we built for the market simply
pulls back along \(j\).

\subsubsection{Carrier extremal maps and the carrier rejection
map}\label{carrier-extremal-maps-and-the-carrier-rejection-map}

Just as the market nucleus admits scalar summaries \((r^*, q^*)\), the
restricted nucleus induces carrier‑specific extremal maps

\[
r^*_{\text{carrier}} : [0,1]_{Q_{\text{carrier}}} \to [0,1]_R,
\qquad
q^*_{\text{carrier}} : [0,1]_R \to [0,1]_{Q_{\text{carrier}}},
\]

defined by

\[
r^*_{\text{carrier}}(q)
 :=
 \max F_{\text{carrier}}^\sharp(\{q\}),
\qquad
q^*_{\text{carrier}}(r)
 :=
 \min F_{\text{carrier}}^\flat(\{r\}).
\]

Unwinding the definitions, the carrier WTP map is simply the market one
composed with the inclusion:

\[
r^*_{\text{carrier}}(q) = r^*\bigl(j(q)\bigr).
\]

For the rejection map, we simply compose the market rejection map with
the left adjoint of the inclusion:

\[
q^*_{\text{carrier}}(r) = j_!\bigl(q^*(r)\bigr).
\]

Here \(r^*_{\text{carrier}}(q)\) is the carrier's structural WTP at
carrier request quantile \(q\), while \(q^*_{\text{carrier}}(r)\) is the
carrier's rejection map: the carrier‑side quantile level at which the
rate \(r\) starts to be rejected.

The carrier's pointwise nucleus mirrors the market one at the level of
extremal points:

\[
\mathrm{Nuc}(F_{\text{carrier}})_{\mathrm{ext}}
= \{(r,q) \in [0,1]_R \times [0,1]_{Q_{\text{carrier}}} : r = r^*_{\text{carrier}}(q),\; q = q^*_{\text{carrier}}(r)\}.
\]

Given this construction, the carrier‑specific rejection probability
curve described in section 8 is obtained by reading
\(q^*_{\text{carrier}}\) as a probability via the same ``quantile = mass
below'' semantics used for the market:

\[
\rho_{\text{carrier}} : [0,1]_R \to [0,1],
\qquad
\rho_{\text{carrier}}(r)
=
\Pr\bigl(\text{rejection at rate } r \mid q \in Q_{\text{carrier}}\bigr)
=
q^*_{\text{carrier}}(r).
\]

Thus the carrier rejection curve is nothing new structurally---it is the
rejection map of the restricted nucleus, evaluated numerically as a
quantile in \([0,1]\).

\section*{Appendix A: Explicit Construction of the Quantile Embedding}

A more explicit construction starts from the canonical feasibility
relation \(\mathrm{hom}_Q : Q^{op}\times Q \to \mathbf{Bool}\), which
just records the order: \(\mathrm{hom}_Q(q',q) = \mathbf{true}\) exactly
when \(q'\le q\). Currying in the first argument, and fixing \(q\), we
read \(\mathrm{hom}_Q(-,q) : Q \to \mathbf{Bool}\) as an ordinary
function. This is the indicator (1 if an element is in a set, 0
otherwise) of the principal down‑set \(\downarrow q\) (all elements
\(\le q\)), written \(\chi_{\downarrow q} : Q \to \mathbf{Bool}\) with

\[
\chi_{\downarrow q}(q') = \mathrm{hom}_Q(q',q).
\]

Averaging those indicators uniformly over all \(q' \in Q\)---summing
the Boolean values (1 when \(q' \le q\), 0 otherwise) and dividing by
\(|Q|\)---produces a cumulative rank. We then rescale so that the
minimum of \(Q\) maps to \(0\) and the maximum to \(1\), with equal
steps in between. This is the same averaging/rescaling used later when
integrating feasibility relations in section 9.2. In terms of the
embedding \(\phi_Q\) defined above, this is

\[
\phi_Q(q)
= \frac{1}{|Q|-1}
\sum_{q' \in Q} \chi_{\downarrow q}(q') - \frac{1}{|Q|-1}
= \frac{1}{|Q|-1}
\sum_{q' \in Q} \mathrm{hom}_Q(q',q) - \frac{1}{|Q|-1}.
\]

Here the sum counts how many elements lie in \(\downarrow q\): each
indicator contributes 1 when \(q'\) is below \(q\) and 0 otherwise.
Subtracting \(1\) and dividing by \(|Q|-1\) linearly rescales this
cumulative rank so that the minimum element of \(Q\) gets value \(0\)
and the maximum gets value \(1\).

\end{document}
